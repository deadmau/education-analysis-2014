
% Default to the notebook output style

    


% Inherit from the specified cell style.




    
\documentclass{article}

    
    
    \usepackage{graphicx} % Used to insert images
    \usepackage{adjustbox} % Used to constrain images to a maximum size 
    \usepackage{color} % Allow colors to be defined
    \usepackage{enumerate} % Needed for markdown enumerations to work
    \usepackage{geometry} % Used to adjust the document margins
    \usepackage{amsmath} % Equations
    \usepackage{amssymb} % Equations
    \usepackage[mathletters,combine]{ucs} % Extended unicode (utf-8) support
    \usepackage[utf8x]{inputenc} % Allow utf-8 characters in the tex document
    \usepackage{fancyvrb} % verbatim replacement that allows latex
    \usepackage{grffile} % extends the file name processing of package graphics 
                         % to support a larger range 
    % The hyperref package gives us a pdf with properly built
    % internal navigation ('pdf bookmarks' for the table of contents,
    % internal cross-reference links, web links for URLs, etc.)
    \usepackage{hyperref}
    \usepackage{longtable} % longtable support required by pandoc >1.10
    

    
    
    \definecolor{orange}{cmyk}{0,0.4,0.8,0.2}
    \definecolor{darkorange}{rgb}{.71,0.21,0.01}
    \definecolor{darkgreen}{rgb}{.12,.54,.11}
    \definecolor{myteal}{rgb}{.26, .44, .56}
    \definecolor{gray}{gray}{0.45}
    \definecolor{lightgray}{gray}{.95}
    \definecolor{mediumgray}{gray}{.8}
    \definecolor{inputbackground}{rgb}{.95, .95, .85}
    \definecolor{outputbackground}{rgb}{.95, .95, .95}
    \definecolor{traceback}{rgb}{1, .95, .95}
    % ansi colors
    \definecolor{red}{rgb}{.6,0,0}
    \definecolor{green}{rgb}{0,.65,0}
    \definecolor{brown}{rgb}{0.6,0.6,0}
    \definecolor{blue}{rgb}{0,.145,.698}
    \definecolor{purple}{rgb}{.698,.145,.698}
    \definecolor{cyan}{rgb}{0,.698,.698}
    \definecolor{lightgray}{gray}{0.5}
    
    % bright ansi colors
    \definecolor{darkgray}{gray}{0.25}
    \definecolor{lightred}{rgb}{1.0,0.39,0.28}
    \definecolor{lightgreen}{rgb}{0.48,0.99,0.0}
    \definecolor{lightblue}{rgb}{0.53,0.81,0.92}
    \definecolor{lightpurple}{rgb}{0.87,0.63,0.87}
    \definecolor{lightcyan}{rgb}{0.5,1.0,0.83}
    
    % commands and environments needed by pandoc snippets
    % extracted from the output of `pandoc -s`
    
    \DefineShortVerb[commandchars=\\\{\}]{\|}
    \DefineVerbatimEnvironment{Highlighting}{Verbatim}{commandchars=\\\{\}}
    % Add ',fontsize=\small' for more characters per line
    \newenvironment{Shaded}{}{}
    \newcommand{\KeywordTok}[1]{\textcolor[rgb]{0.00,0.44,0.13}{\textbf{{#1}}}}
    \newcommand{\DataTypeTok}[1]{\textcolor[rgb]{0.56,0.13,0.00}{{#1}}}
    \newcommand{\DecValTok}[1]{\textcolor[rgb]{0.25,0.63,0.44}{{#1}}}
    \newcommand{\BaseNTok}[1]{\textcolor[rgb]{0.25,0.63,0.44}{{#1}}}
    \newcommand{\FloatTok}[1]{\textcolor[rgb]{0.25,0.63,0.44}{{#1}}}
    \newcommand{\CharTok}[1]{\textcolor[rgb]{0.25,0.44,0.63}{{#1}}}
    \newcommand{\StringTok}[1]{\textcolor[rgb]{0.25,0.44,0.63}{{#1}}}
    \newcommand{\CommentTok}[1]{\textcolor[rgb]{0.38,0.63,0.69}{\textit{{#1}}}}
    \newcommand{\OtherTok}[1]{\textcolor[rgb]{0.00,0.44,0.13}{{#1}}}
    \newcommand{\AlertTok}[1]{\textcolor[rgb]{1.00,0.00,0.00}{\textbf{{#1}}}}
    \newcommand{\FunctionTok}[1]{\textcolor[rgb]{0.02,0.16,0.49}{{#1}}}
    \newcommand{\RegionMarkerTok}[1]{{#1}}
    \newcommand{\ErrorTok}[1]{\textcolor[rgb]{1.00,0.00,0.00}{\textbf{{#1}}}}
    \newcommand{\NormalTok}[1]{{#1}}
    
    % Define a nice break command that doesn't care if a line doesn't already
    % exist.
    \def\br{\hspace*{\fill} \\* }
    % Math Jax compatability definitions
    \def\gt{>}
    \def\lt{<}
    % Document parameters
    \title{NB4\_project\_report}
    
    
    

    % Pygments definitions
    
\makeatletter
\def\PY@reset{\let\PY@it=\relax \let\PY@bf=\relax%
    \let\PY@ul=\relax \let\PY@tc=\relax%
    \let\PY@bc=\relax \let\PY@ff=\relax}
\def\PY@tok#1{\csname PY@tok@#1\endcsname}
\def\PY@toks#1+{\ifx\relax#1\empty\else%
    \PY@tok{#1}\expandafter\PY@toks\fi}
\def\PY@do#1{\PY@bc{\PY@tc{\PY@ul{%
    \PY@it{\PY@bf{\PY@ff{#1}}}}}}}
\def\PY#1#2{\PY@reset\PY@toks#1+\relax+\PY@do{#2}}

\expandafter\def\csname PY@tok@gd\endcsname{\def\PY@tc##1{\textcolor[rgb]{0.63,0.00,0.00}{##1}}}
\expandafter\def\csname PY@tok@gu\endcsname{\let\PY@bf=\textbf\def\PY@tc##1{\textcolor[rgb]{0.50,0.00,0.50}{##1}}}
\expandafter\def\csname PY@tok@gt\endcsname{\def\PY@tc##1{\textcolor[rgb]{0.00,0.27,0.87}{##1}}}
\expandafter\def\csname PY@tok@gs\endcsname{\let\PY@bf=\textbf}
\expandafter\def\csname PY@tok@gr\endcsname{\def\PY@tc##1{\textcolor[rgb]{1.00,0.00,0.00}{##1}}}
\expandafter\def\csname PY@tok@cm\endcsname{\let\PY@it=\textit\def\PY@tc##1{\textcolor[rgb]{0.25,0.50,0.50}{##1}}}
\expandafter\def\csname PY@tok@vg\endcsname{\def\PY@tc##1{\textcolor[rgb]{0.10,0.09,0.49}{##1}}}
\expandafter\def\csname PY@tok@m\endcsname{\def\PY@tc##1{\textcolor[rgb]{0.40,0.40,0.40}{##1}}}
\expandafter\def\csname PY@tok@mh\endcsname{\def\PY@tc##1{\textcolor[rgb]{0.40,0.40,0.40}{##1}}}
\expandafter\def\csname PY@tok@go\endcsname{\def\PY@tc##1{\textcolor[rgb]{0.53,0.53,0.53}{##1}}}
\expandafter\def\csname PY@tok@ge\endcsname{\let\PY@it=\textit}
\expandafter\def\csname PY@tok@vc\endcsname{\def\PY@tc##1{\textcolor[rgb]{0.10,0.09,0.49}{##1}}}
\expandafter\def\csname PY@tok@il\endcsname{\def\PY@tc##1{\textcolor[rgb]{0.40,0.40,0.40}{##1}}}
\expandafter\def\csname PY@tok@cs\endcsname{\let\PY@it=\textit\def\PY@tc##1{\textcolor[rgb]{0.25,0.50,0.50}{##1}}}
\expandafter\def\csname PY@tok@cp\endcsname{\def\PY@tc##1{\textcolor[rgb]{0.74,0.48,0.00}{##1}}}
\expandafter\def\csname PY@tok@gi\endcsname{\def\PY@tc##1{\textcolor[rgb]{0.00,0.63,0.00}{##1}}}
\expandafter\def\csname PY@tok@gh\endcsname{\let\PY@bf=\textbf\def\PY@tc##1{\textcolor[rgb]{0.00,0.00,0.50}{##1}}}
\expandafter\def\csname PY@tok@ni\endcsname{\let\PY@bf=\textbf\def\PY@tc##1{\textcolor[rgb]{0.60,0.60,0.60}{##1}}}
\expandafter\def\csname PY@tok@nl\endcsname{\def\PY@tc##1{\textcolor[rgb]{0.63,0.63,0.00}{##1}}}
\expandafter\def\csname PY@tok@nn\endcsname{\let\PY@bf=\textbf\def\PY@tc##1{\textcolor[rgb]{0.00,0.00,1.00}{##1}}}
\expandafter\def\csname PY@tok@no\endcsname{\def\PY@tc##1{\textcolor[rgb]{0.53,0.00,0.00}{##1}}}
\expandafter\def\csname PY@tok@na\endcsname{\def\PY@tc##1{\textcolor[rgb]{0.49,0.56,0.16}{##1}}}
\expandafter\def\csname PY@tok@nb\endcsname{\def\PY@tc##1{\textcolor[rgb]{0.00,0.50,0.00}{##1}}}
\expandafter\def\csname PY@tok@nc\endcsname{\let\PY@bf=\textbf\def\PY@tc##1{\textcolor[rgb]{0.00,0.00,1.00}{##1}}}
\expandafter\def\csname PY@tok@nd\endcsname{\def\PY@tc##1{\textcolor[rgb]{0.67,0.13,1.00}{##1}}}
\expandafter\def\csname PY@tok@ne\endcsname{\let\PY@bf=\textbf\def\PY@tc##1{\textcolor[rgb]{0.82,0.25,0.23}{##1}}}
\expandafter\def\csname PY@tok@nf\endcsname{\def\PY@tc##1{\textcolor[rgb]{0.00,0.00,1.00}{##1}}}
\expandafter\def\csname PY@tok@si\endcsname{\let\PY@bf=\textbf\def\PY@tc##1{\textcolor[rgb]{0.73,0.40,0.53}{##1}}}
\expandafter\def\csname PY@tok@s2\endcsname{\def\PY@tc##1{\textcolor[rgb]{0.73,0.13,0.13}{##1}}}
\expandafter\def\csname PY@tok@vi\endcsname{\def\PY@tc##1{\textcolor[rgb]{0.10,0.09,0.49}{##1}}}
\expandafter\def\csname PY@tok@nt\endcsname{\let\PY@bf=\textbf\def\PY@tc##1{\textcolor[rgb]{0.00,0.50,0.00}{##1}}}
\expandafter\def\csname PY@tok@nv\endcsname{\def\PY@tc##1{\textcolor[rgb]{0.10,0.09,0.49}{##1}}}
\expandafter\def\csname PY@tok@s1\endcsname{\def\PY@tc##1{\textcolor[rgb]{0.73,0.13,0.13}{##1}}}
\expandafter\def\csname PY@tok@sh\endcsname{\def\PY@tc##1{\textcolor[rgb]{0.73,0.13,0.13}{##1}}}
\expandafter\def\csname PY@tok@sc\endcsname{\def\PY@tc##1{\textcolor[rgb]{0.73,0.13,0.13}{##1}}}
\expandafter\def\csname PY@tok@sx\endcsname{\def\PY@tc##1{\textcolor[rgb]{0.00,0.50,0.00}{##1}}}
\expandafter\def\csname PY@tok@bp\endcsname{\def\PY@tc##1{\textcolor[rgb]{0.00,0.50,0.00}{##1}}}
\expandafter\def\csname PY@tok@c1\endcsname{\let\PY@it=\textit\def\PY@tc##1{\textcolor[rgb]{0.25,0.50,0.50}{##1}}}
\expandafter\def\csname PY@tok@kc\endcsname{\let\PY@bf=\textbf\def\PY@tc##1{\textcolor[rgb]{0.00,0.50,0.00}{##1}}}
\expandafter\def\csname PY@tok@c\endcsname{\let\PY@it=\textit\def\PY@tc##1{\textcolor[rgb]{0.25,0.50,0.50}{##1}}}
\expandafter\def\csname PY@tok@mf\endcsname{\def\PY@tc##1{\textcolor[rgb]{0.40,0.40,0.40}{##1}}}
\expandafter\def\csname PY@tok@err\endcsname{\def\PY@bc##1{\setlength{\fboxsep}{0pt}\fcolorbox[rgb]{1.00,0.00,0.00}{1,1,1}{\strut ##1}}}
\expandafter\def\csname PY@tok@kd\endcsname{\let\PY@bf=\textbf\def\PY@tc##1{\textcolor[rgb]{0.00,0.50,0.00}{##1}}}
\expandafter\def\csname PY@tok@ss\endcsname{\def\PY@tc##1{\textcolor[rgb]{0.10,0.09,0.49}{##1}}}
\expandafter\def\csname PY@tok@sr\endcsname{\def\PY@tc##1{\textcolor[rgb]{0.73,0.40,0.53}{##1}}}
\expandafter\def\csname PY@tok@mo\endcsname{\def\PY@tc##1{\textcolor[rgb]{0.40,0.40,0.40}{##1}}}
\expandafter\def\csname PY@tok@kn\endcsname{\let\PY@bf=\textbf\def\PY@tc##1{\textcolor[rgb]{0.00,0.50,0.00}{##1}}}
\expandafter\def\csname PY@tok@mi\endcsname{\def\PY@tc##1{\textcolor[rgb]{0.40,0.40,0.40}{##1}}}
\expandafter\def\csname PY@tok@gp\endcsname{\let\PY@bf=\textbf\def\PY@tc##1{\textcolor[rgb]{0.00,0.00,0.50}{##1}}}
\expandafter\def\csname PY@tok@o\endcsname{\def\PY@tc##1{\textcolor[rgb]{0.40,0.40,0.40}{##1}}}
\expandafter\def\csname PY@tok@kr\endcsname{\let\PY@bf=\textbf\def\PY@tc##1{\textcolor[rgb]{0.00,0.50,0.00}{##1}}}
\expandafter\def\csname PY@tok@s\endcsname{\def\PY@tc##1{\textcolor[rgb]{0.73,0.13,0.13}{##1}}}
\expandafter\def\csname PY@tok@kp\endcsname{\def\PY@tc##1{\textcolor[rgb]{0.00,0.50,0.00}{##1}}}
\expandafter\def\csname PY@tok@w\endcsname{\def\PY@tc##1{\textcolor[rgb]{0.73,0.73,0.73}{##1}}}
\expandafter\def\csname PY@tok@kt\endcsname{\def\PY@tc##1{\textcolor[rgb]{0.69,0.00,0.25}{##1}}}
\expandafter\def\csname PY@tok@ow\endcsname{\let\PY@bf=\textbf\def\PY@tc##1{\textcolor[rgb]{0.67,0.13,1.00}{##1}}}
\expandafter\def\csname PY@tok@sb\endcsname{\def\PY@tc##1{\textcolor[rgb]{0.73,0.13,0.13}{##1}}}
\expandafter\def\csname PY@tok@k\endcsname{\let\PY@bf=\textbf\def\PY@tc##1{\textcolor[rgb]{0.00,0.50,0.00}{##1}}}
\expandafter\def\csname PY@tok@se\endcsname{\let\PY@bf=\textbf\def\PY@tc##1{\textcolor[rgb]{0.73,0.40,0.13}{##1}}}
\expandafter\def\csname PY@tok@sd\endcsname{\let\PY@it=\textit\def\PY@tc##1{\textcolor[rgb]{0.73,0.13,0.13}{##1}}}

\def\PYZbs{\char`\\}
\def\PYZus{\char`\_}
\def\PYZob{\char`\{}
\def\PYZcb{\char`\}}
\def\PYZca{\char`\^}
\def\PYZam{\char`\&}
\def\PYZlt{\char`\<}
\def\PYZgt{\char`\>}
\def\PYZsh{\char`\#}
\def\PYZpc{\char`\%}
\def\PYZdl{\char`\$}
\def\PYZhy{\char`\-}
\def\PYZsq{\char`\'}
\def\PYZdq{\char`\"}
\def\PYZti{\char`\~}
% for compatibility with earlier versions
\def\PYZat{@}
\def\PYZlb{[}
\def\PYZrb{]}
\makeatother


    % Exact colors from NB
    \definecolor{incolor}{rgb}{0.0, 0.0, 0.5}
    \definecolor{outcolor}{rgb}{0.545, 0.0, 0.0}



    
    % Prevent overflowing lines due to hard-to-break entities
    \sloppy 
    % Setup hyperref package
    \hypersetup{
      breaklinks=true,  % so long urls are correctly broken across lines
      colorlinks=true,
      urlcolor=blue,
      linkcolor=darkorange,
      citecolor=darkgreen,
      }
    % Slightly bigger margins than the latex defaults
    
    \geometry{verbose,tmargin=1in,bmargin=1in,lmargin=1in,rmargin=1in}
    
    

    \begin{document}
    
    
    \maketitle
    
    

    
    In this notebook, you'll sumarize your finding for the class
presentation.

The class presentation will use this notebook in slide mode, as I do
during lecture time.

You'll need to download the following
\href{https://github.com/fperez/nb-slideshow-template/archive/master.zip}{zip
file}, unzip it, and run the following notebook install-support.ipynb to
install the slide capabilities.

The notebook notebook-slideshow-example.ipynb will give you examples on
how to use slides within the iPython notebook.

You should also write this notebook so that you can convert it nicely
into a pdf document using the commands

\begin{verbatim}
ipython nbconvert NB4_report.pynb --to latex
pdflatex NB4_report.tex
\end{verbatim}

See
\href{http://ipython.org/ipython-doc/rel-1.0.0/interactive/nbconvert.html}{here}
for further references on how to do that.

In this notebook, you'll

\begin{itemize}
\item
  describe your problem as stated in the propectus
\item
  comment on your data sources, on their format, on the difficulties to
  get them
\item
  present the main challenge you encountered
\item
  present your finding in the form of expresive graphics
\end{itemize}

Be sure to include an introduction section motivating your
visualizations, with a description of the substantive context and why it
is interesting.

Cite the source of the data and any other references that you used in
carrying out your project.

    \subsection{Team members responsible for this
notebook:}\label{team-members-responsible-for-this-notebook}

\begin{itemize}
\itemsep1pt\parskip0pt\parsep0pt
\item
  team member 1 \textbf{Shadman Sadek}: Compiling and completing this
  notebook
\item
  team member 2 \textbf{Elizabeth Sabiniano}: Compiling and completing
  this notebook
\end{itemize}


    \section{Objective:}


    The purpose of this project is to examine how sentiment towards college
varies throughout the world - both between continents (general
geographical and socioeconomic areas) (and between states specifically
in the US?) We used Python and Twitter API to help gather a massive
amount of data (\textasciitilde{}500,000 tweets) based on the keyword
\textbf{college}. The data gathered was organized into tweets with
location and without location. Furthermore, tweets with location were
further categorized based on global location.

Our data was then analyzed into positive, negative, and neutral
sentiment categories. We developed visualizations that displayed word
frequencies and a map that displays sentiment based on global region.


    \subsection{Data Sources:}


    All of our data was obtained from Twitter. These were then saved in
pickle files to accomodate for the large datastrings.


    \subsection{Challenges:}


    Originally, our team wanted to examine the sentiments towards higher
education, however the keywords ``higher education'' did not yield
enough data for us to work with. That and the following are the
challenges we faced in this study:

\begin{itemize}
\itemsep1pt\parskip0pt\parsep0pt
\item
  Finding the right program to gather enough tweets.
\item
  Gathering enough data with location
\item
  Machine learning: being able to train our program to categorize as
  much of the tweets as accurately as possible
\end{itemize}

We tried TwitterSearch, Tweepy, before finally finding Twython, which
yielded enough data for us to work with.


    \subsection{Visualizations: Word Frequency}


    The following visualizations describe word frequencies in Asia, Africa,
Europe, Latin America, United States, and the world. They were obtained
through the word cloud program called Wordle. The most frequent words
show up largest in the visualizations. Such vizualizations give us an
idea of topics most commonly associated with our keyword
\textbf{college}.

    \begin{Verbatim}[commandchars=\\\{\}]
{\color{incolor}In [{\color{incolor}1}]:} \PY{k+kn}{from} \PY{n+nn}{IPython.core.display} \PY{k+kn}{import} \PY{n}{Image} 
\end{Verbatim}


    \paragraph{Asia}


    \begin{Verbatim}[commandchars=\\\{\}]
{\color{incolor}In [{\color{incolor}17}]:} \PY{n}{Image}\PY{p}{(}\PY{n}{filename}\PY{o}{=}\PY{l+s}{\PYZsq{}}\PY{l+s}{../visualizations/AsiaWords.png}\PY{l+s}{\PYZsq{}}\PY{p}{)}
\end{Verbatim}
\texttt{\color{outcolor}Out[{\color{outcolor}17}]:}
    
    \begin{center}
    \adjustimage{max size={0.9\linewidth}{0.9\paperheight}}{NB4_project_report_files/NB4_project_report_12_0.png}
    \end{center}
    { \hspace*{\fill} \\}
    

    \emph{jateng} stands for Jateng-DIY, a province in Indonesia. Over the
period at which we gathered our data, the ESA Week 2014 (a scholastic
competition) is taking place in this region.


    \paragraph{Africa}


    \begin{Verbatim}[commandchars=\\\{\}]
{\color{incolor}In [{\color{incolor}4}]:} \PY{n}{Image}\PY{p}{(}\PY{n}{filename}\PY{o}{=}\PY{l+s}{\PYZsq{}}\PY{l+s}{../visualizations/AfricaWord.png}\PY{l+s}{\PYZsq{}}\PY{p}{)}
\end{Verbatim}
\texttt{\color{outcolor}Out[{\color{outcolor}4}]:}
    
    \begin{center}
    \adjustimage{max size={0.9\linewidth}{0.9\paperheight}}{NB4_project_report_files/NB4_project_report_15_0.png}
    \end{center}
    { \hspace*{\fill} \\}
    

    Each of the words above appeared with almost equal amount of frequency;
thus, they are all close in size. \emph{shameemah} is a user's name who
has a decent amount of followers and obtained several retweets regarding
college inquiries. \emph{best} appears to be the most frequent word in
this location.


    \paragraph{Europe}


    \begin{Verbatim}[commandchars=\\\{\}]
{\color{incolor}In [{\color{incolor}8}]:} \PY{n}{Image}\PY{p}{(}\PY{n}{filename}\PY{o}{=}\PY{l+s}{\PYZsq{}}\PY{l+s}{../visualizations/EuropeWords.png}\PY{l+s}{\PYZsq{}}\PY{p}{)}
\end{Verbatim}
\texttt{\color{outcolor}Out[{\color{outcolor}8}]:}
    
    \begin{center}
    \adjustimage{max size={0.9\linewidth}{0.9\paperheight}}{NB4_project_report_files/NB4_project_report_18_0.png}
    \end{center}
    { \hspace*{\fill} \\}
    

    Although words such as \emph{good} and \emph{last} appear the most
frequent in this region, the tweets at which they appear are mostly
negations. For example:

\begin{itemize}
\itemsep1pt\parskip0pt\parsep0pt
\item
  ``Can't wait to finish college for good. 6 weeks left''
\item
  `Not sure college is a good idea today'
\item
  ``It's safe to say I haven't missed college one bit these last 2 weeks
  don't wannnna go back''
\item
  `so much to do, zero motivation for the last week of college, this is
  not good..'
\end{itemize}


    \paragraph{Latin America}


    \begin{Verbatim}[commandchars=\\\{\}]
{\color{incolor}In [{\color{incolor}10}]:} \PY{n}{Image}\PY{p}{(}\PY{n}{filename}\PY{o}{=}\PY{l+s}{\PYZsq{}}\PY{l+s}{../visualizations/LatinAmericaWords.png}\PY{l+s}{\PYZsq{}}\PY{p}{)}
\end{Verbatim}
\texttt{\color{outcolor}Out[{\color{outcolor}10}]:}
    
    \begin{center}
    \adjustimage{max size={0.9\linewidth}{0.9\paperheight}}{NB4_project_report_files/NB4_project_report_21_0.png}
    \end{center}
    { \hspace*{\fill} \\}
    

    Similar to Africa, most the words listed for Latin America have the same
frequency. The bigger words are only one or two counts away from the
smaller ones.


    \paragraph{United States}


    \begin{Verbatim}[commandchars=\\\{\}]
{\color{incolor}In [{\color{incolor}2}]:} \PY{n}{Image}\PY{p}{(}\PY{n}{filename}\PY{o}{=}\PY{l+s}{\PYZsq{}}\PY{l+s}{../visualizations/USWords.png}\PY{l+s}{\PYZsq{}}\PY{p}{)}
\end{Verbatim}
\texttt{\color{outcolor}Out[{\color{outcolor}2}]:}
    
    \begin{center}
    \adjustimage{max size={0.9\linewidth}{0.9\paperheight}}{NB4_project_report_files/NB4_project_report_24_0.png}
    \end{center}
    { \hspace*{\fill} \\}
    

    Most of the tweets obtained from the US either talk about the user's
first year of college or the last days they will spend as a college
student.


    \paragraph{World}


    \begin{Verbatim}[commandchars=\\\{\}]
{\color{incolor}In [{\color{incolor}3}]:} \PY{n}{Image}\PY{p}{(}\PY{n}{filename}\PY{o}{=}\PY{l+s}{\PYZsq{}}\PY{l+s}{../visualizations/WorldWords.png}\PY{l+s}{\PYZsq{}}\PY{p}{)}
\end{Verbatim}
\texttt{\color{outcolor}Out[{\color{outcolor}3}]:}
    
    \begin{center}
    \adjustimage{max size={0.9\linewidth}{0.9\paperheight}}{NB4_project_report_files/NB4_project_report_27_0.png}
    \end{center}
    { \hspace*{\fill} \\}
    

    This shows all the frequent words across the locations from which we
gathered our data.


    \subsubsection{Visualizations: Global Sentiment}


    The following visualizations show differences in sentiment based on the
keyword \textbf{college} by global region. These visualizations give us
an idea of differing sentiment based on location and can help in making
determinations on why there is differing sentiment if any (demographics?
political?, socioeconomics?).


    \paragraph{Asian Sentiment}


    \begin{Verbatim}[commandchars=\\\{\}]
{\color{incolor}In [{\color{incolor}6}]:} \PY{n}{Image}\PY{p}{(}\PY{n}{filename}\PY{o}{=}\PY{l+s}{\PYZsq{}}\PY{l+s}{../visualizations/Asia.png}\PY{l+s}{\PYZsq{}}\PY{p}{)}
\end{Verbatim}
\texttt{\color{outcolor}Out[{\color{outcolor}6}]:}
    
    \begin{center}
    \adjustimage{max size={0.9\linewidth}{0.9\paperheight}}{NB4_project_report_files/NB4_project_report_32_0.png}
    \end{center}
    { \hspace*{\fill} \\}
    


    \paragraph{African Sentiment}


    \begin{Verbatim}[commandchars=\\\{\}]
{\color{incolor}In [{\color{incolor}19}]:} \PY{n}{Image}\PY{p}{(}\PY{n}{filename}\PY{o}{=}\PY{l+s}{\PYZsq{}}\PY{l+s}{../visualizations/Africa.png}\PY{l+s}{\PYZsq{}}\PY{p}{)}
\end{Verbatim}
\texttt{\color{outcolor}Out[{\color{outcolor}19}]:}
    
    \begin{center}
    \adjustimage{max size={0.9\linewidth}{0.9\paperheight}}{NB4_project_report_files/NB4_project_report_34_0.png}
    \end{center}
    { \hspace*{\fill} \\}
    


    \paragraph{European Sentiment}


    \begin{Verbatim}[commandchars=\\\{\}]
{\color{incolor}In [{\color{incolor}20}]:} \PY{n}{Image}\PY{p}{(}\PY{n}{filename}\PY{o}{=}\PY{l+s}{\PYZsq{}}\PY{l+s}{../visualizations/Europe.png}\PY{l+s}{\PYZsq{}}\PY{p}{)}
\end{Verbatim}
\texttt{\color{outcolor}Out[{\color{outcolor}20}]:}
    
    \begin{center}
    \adjustimage{max size={0.9\linewidth}{0.9\paperheight}}{NB4_project_report_files/NB4_project_report_36_0.png}
    \end{center}
    { \hspace*{\fill} \\}
    


    \paragraph{Latin American Sentiment}


    \begin{Verbatim}[commandchars=\\\{\}]
{\color{incolor}In [{\color{incolor}21}]:} \PY{n}{Image}\PY{p}{(}\PY{n}{filename}\PY{o}{=}\PY{l+s}{\PYZsq{}}\PY{l+s}{../visualizations/Latin America.png}\PY{l+s}{\PYZsq{}}\PY{p}{)}
\end{Verbatim}
\texttt{\color{outcolor}Out[{\color{outcolor}21}]:}
    
    \begin{center}
    \adjustimage{max size={0.9\linewidth}{0.9\paperheight}}{NB4_project_report_files/NB4_project_report_38_0.png}
    \end{center}
    { \hspace*{\fill} \\}
    


    \paragraph{American Sentiment}


    \begin{Verbatim}[commandchars=\\\{\}]
{\color{incolor}In [{\color{incolor}22}]:} \PY{n}{Image}\PY{p}{(}\PY{n}{filename}\PY{o}{=}\PY{l+s}{\PYZsq{}}\PY{l+s}{../visualizations/United States.png}\PY{l+s}{\PYZsq{}}\PY{p}{)}
\end{Verbatim}
\texttt{\color{outcolor}Out[{\color{outcolor}22}]:}
    
    \begin{center}
    \adjustimage{max size={0.9\linewidth}{0.9\paperheight}}{NB4_project_report_files/NB4_project_report_40_0.png}
    \end{center}
    { \hspace*{\fill} \\}
    


    \paragraph{World Sentiment}


    \begin{Verbatim}[commandchars=\\\{\}]
{\color{incolor}In [{\color{incolor}24}]:} \PY{n}{Image}\PY{p}{(}\PY{n}{filename}\PY{o}{=}\PY{l+s}{\PYZsq{}}\PY{l+s}{../visualizations/World.png}\PY{l+s}{\PYZsq{}}\PY{p}{)} 
\end{Verbatim}
\texttt{\color{outcolor}Out[{\color{outcolor}24}]:}
    
    \begin{center}
    \adjustimage{max size={0.9\linewidth}{0.9\paperheight}}{NB4_project_report_files/NB4_project_report_42_0.png}
    \end{center}
    { \hspace*{\fill} \\}
    


    \subsection{Conclusions:}


    Overall, we see a global sentiment of negativity towards
\textbf{college}. All of the global regions besides Latin America vastly
display a negative sentiment. Although in Latin America, the sample size
is much smaller so it is difficult to make conclusive remarks on the
level of negativity. These results suggest that there is a universal
negative feeling towards college regardless of location, socioeconomic
status, and culture.

According to this year's Twitter study of Beevolve, 73.7\% of Twitter
users are of age 15-25, where more than half of its total users live in
the United States (with 50.99\%). Twitter has not yet reached rural
areas, thus we could assume that the areas identified in this research
are mainly from highly industrialized areas of the world. Due to the
dominance of 15-25 year-old users, we can also presume that most are in
or on their way to college. There are many factors as to why our results
mainly lead to negative sentiments towards college. Given that these
data were obtained in a 2-week period (last two weeks of April), the
tweets could have been highly influenced by finals/graduation/college
application/acceptance, etc. During this period, the other side of the
world, such as Asia, is undergoing final examinations and extreme
pressure to do well.

Additionally, the words \textbf{first}, \textbf{high}, and \textbf{last}
appeared the most in the respective regions analyzed above. These words
may well explain the sentiment most users (gathered in this research)
feel towards college. Examples from various locations include:

\begin{itemize}
\itemsep1pt\parskip0pt\parsep0pt
\item
  ``@EssentialFact: A study has found that the first two years of
  college are basically useless.''\n\nthat really sums up everything.'
  \emph{(Asia)}
\item
  `Sila ayos na college papers and files tas June-ish pa classes nila,
  ako last week of May na waley pa rin na aaccomplish. DLSU pls. Huhuhu'
  \emph{(Asia)}
\item
  ``First I don't get to move Jaki into college and now I find out that
  we don't have the same spring break. FUCK EVERYTHING'' (\emph{US)}
\item
  `In an email to extended family my dad just referred to my school as
  ``clown college'' and discredited everything I've done over last 4
  years.' \emph{(US)}
\item
  `first day back at college tomorrow how am i going to do this help
  me.' \emph{(Europe)}
\item
  'Last week in college poses a big struggle.. So much work to do but so
  much going out to do at the same time. \emph{(Europe)}
\item
  `Things i never learned in high school: how to: pay bills buy a house
  apply for college but thank god i can graph a polynomial function'
  \emph{(Africa)}
\end{itemize}

Seeing that the sample size for some of the sentiment analysis are very
low, users in this region can cause some bias in the analysis. Users
could also tweet as often as they want about the same topic for a
certain length of time, such as college. The sentiments, which we have
gathered, may be highly obtained from a few sample of users of whom
expressed the same sentiment about college for some amount of time. We
also have to consider that some of the tweets obtained are mixed with
English and some other languages, including but not limited to Spanish,
Indonesian, Filipino. Words written in these languages will not be
classified by the Classifier (refer to ../NB3\_data\_analysis for the
sentiment analysis). The foreign words may have been suggesting positive
sentiments, but our Classifier is limited to the English language.

Therefore, we cannot fully discern whether other factors, such as
socioeconomic status, cultural tradition of the country, or their
current standards of education, play the most significant roles in the
data gathered. However, we could conclude that based on these recent
tweets, users feel an overarching negativity towards college due to a
some of the following reasons: high amount of work, acclamation from
family, college culture shock after Sprin Break, etc.


    % Add a bibliography block to the postdoc
    
    
    
    \end{document}
